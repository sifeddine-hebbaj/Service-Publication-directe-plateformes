\documentclass[a4paper,12pt]{report}
\usepackage[utf8]{inputenc}
\usepackage[T1]{fontenc}
\usepackage[french]{babel}
\usepackage{graphicx}
\usepackage{geometry}
\usepackage{hyperref}
\usepackage{caption}
\usepackage{fancybox}
\usepackage{tikz}
\usepackage{xcolor}
\usepackage{setspace}
\usepackage{emptypage}
\usepackage{enumitem}
\usepackage{float}

\geometry{margin=2.5cm}
\captionsetup[figure]{labelfont=bf}

% Style de la page de titre
\newcommand{\titlerule}[1][1pt]{\rule{\linewidth}{#1}}

\begin{document}

% ==== PAGE DE GARDE ====
\begin{titlepage}
    \thispagestyle{empty}
    \begin{tikzpicture}[remember picture,overlay]
        % Cadre gauche
        \node[draw, thick, minimum width=0.3\textwidth, minimum height=3cm, anchor=north west] 
            at ([xshift=2.5cm, yshift=-2.5cm]current page.north west) 
            (leftbox) {
                \begin{minipage}{0.25\textwidth}
                    \centering
                    \includegraphics[width=2cm]{LogoFsr (1).png} \\
                    \vspace{0.3cm}
                    \textbf{\small Université Mohammed V} \\
                    \textbf{\scriptsize Faculté des Sciences Rabat} 
                \end{minipage}
            };
        
        % Cadre droit
        \node[draw, thick, minimum width=0.3\textwidth, minimum height=3cm, anchor=north east] 
            at ([xshift=-2.5cm, yshift=-2.5cm]current page.north east) 
            (rightbox) {
                \begin{minipage}{0.25\textwidth}
                    \centering
                    \includegraphics[width=2cm]{nexotek.png} \\
                    \vspace{0.3cm}
                    \textbf{\small Nexotek} 
                \end{minipage}
            };
    \end{tikzpicture}

    \vspace*{4cm} % Espacement depuis le haut
    
    \begin{center}
        % Titre principal
        {\Huge \bfseries SERVICE DE PUBLICATION DIRECTE \\ D'OFFRES SUR LINKEDIN} \\[1cm]
        \titlerule[1.5pt] \\[1cm]
        {\LARGE \bfseries Rapport Technique} \\[1cm]
        \titlerule[1.5pt] \\
        
        \vspace{3cm}
        
        % Informations complémentaires
        \begin{minipage}{0.8\textwidth}
            \centering
            \large
            \textbf{Réalisé par:} \\
            \vspace{0.3cm}
            \textbf{HEBBAJ Sif-Eddine} \\
            \vspace{1cm}
            \textbf{Encadré par:} \\
            \vspace{0.3cm}
            \textbf{Pr. Zakaria azelmadf} \\
            \vspace{1cm}
            \textbf{Année universitaire:} \\
            \vspace{0.3cm}
            \textbf{2023-2024}
        \end{minipage}
        
        \vfill
        
        % Date
        {\large \today}
    \end{center}
\end{titlepage}

% ==== TABLE DES MATIÈRES ====
\tableofcontents
\clearpage

% ==== INTRODUCTION ====
\chapter*{Introduction}
\addcontentsline{toc}{chapter}{Introduction}
\thispagestyle{plain}
Ce rapport présente l'architecture, le fonctionnement et les principaux flux du service de publication directe d'offres d'emploi sur LinkedIn, développé en Java avec Spring Boot. Ce service permet à un recruteur de publier une offre sur LinkedIn via une API REST, tout en stockant l'offre dans une base de données PostgreSQL.

\chapter{Architecture Générale}
\section{Structure du Service}
Le service est structuré selon une architecture en couches typique des applications Spring Boot:

\begin{itemize}[leftmargin=*]
    \item \textbf{Controller}: Point d'entrée REST, reçoit les requêtes HTTP
    \item \textbf{Service}: Logique métier, gestion de la publication et du stockage
    \item \textbf{Mapper/DTO/Entity}: Conversion et modélisation des données
    \item \textbf{Repository}: Accès à la base de données
\end{itemize}

\section{Diagramme de Classes}
\begin{figure}[H]
    \centering
    \includegraphics[width=0.95\textwidth]{dc.png}
    \caption{Diagramme de classes du service}
    \label{fig:class_diagram}
\end{figure}

\textbf{Explication:}
\begin{itemize}[leftmargin=*]
    \item \texttt{JobPublicationController} reçoit la requête de publication et délègue au service
    \item \texttt{JobPublicationService} gère la logique de publication sur LinkedIn et le stockage
    \item \texttt{JobOfferMapper} convertit entre DTO et entités
    \item \texttt{JobOfferRepository} permet la persistance des offres
    \item \texttt{JobOffer}, \texttt{JobOfferRequestDTO}, \texttt{JobOfferResponseDTO} modélisent les données
\end{itemize}

\chapter{Flux Fonctionnel}
\section{Diagramme de Cas d'Usage}
\begin{figure}[H]
    \centering
    \includegraphics[width=0.85\textwidth]{UC.png}
    \caption{Diagramme de cas d'usage}
    \label{fig:use_case}
\end{figure}

\textbf{Explication:}
\begin{itemize}[leftmargin=*]
    \item Le \textbf{recruteur} initie la publication d'une offre
    \item Le service valide le token LinkedIn, construit le texte du post, publie sur LinkedIn, gère les erreurs et stocke l'offre en base
    \item L'API LinkedIn et la base PostgreSQL sont des systèmes externes
\end{itemize}

\section{Diagramme de Transition d'État}
\begin{figure}[H]
    \centering
    \includegraphics[width=0.7\textwidth]{Untitled diagram _ Mermaid Chart-2025-07-13-152612.png}
    \caption{Diagramme de transition d'état du processus de publication}
    \label{fig:state_diagram}
\end{figure}

\textbf{Explication détaillée:}
\begin{enumerate}[leftmargin=*]
    \item \textbf{EnAttente}: Le service attend une requête
    \item \textbf{ValidationToken}: Vérification du token LinkedIn
    \item \textbf{TokenValide/TokenInvalide}: Si le token est valide, on continue, sinon on retourne une erreur
    \item \textbf{ConstructionPost}: Génération du texte LinkedIn
    \item \textbf{PublicationLinkedIn}: Tentative de publication sur LinkedIn
    \item \textbf{PublicationSucces/PublicationEchec}: Selon la réponse de LinkedIn
    \item \textbf{SauvegardeBase}: Stockage de l'offre en base
    \item \textbf{OffreStockee}: Réponse envoyée au client
\end{enumerate}

\chapter{Détail du Fonctionnement}
\section{Étapes principales}
\begin{enumerate}[leftmargin=*]
    \item Le recruteur envoie une requête POST avec les détails de l'offre et un token LinkedIn
    \item Le contrôleur extrait et valide le token
    \item Le service construit le texte du post et tente la publication via l'API LinkedIn
    \item Quel que soit le résultat, l'offre est stockée en base avec le texte généré
    \item Le service retourne un message de succès ou d'erreur
\end{enumerate}

\section{Gestion des erreurs}
\begin{itemize}[leftmargin=*]
    \item Si le token est manquant ou invalide, une erreur explicite est retournée
    \item Si la publication LinkedIn échoue, l'offre est tout de même stockée et l'erreur est signalée
\end{itemize}

\chapter{Spécification Fonctionnelle Détaillée}

\section{Cas d'utilisation principal}
\textbf{UC1 : Publier une offre sur LinkedIn}
\begin{itemize}[leftmargin=*]
    \item \textbf{Acteur principal} : Recruteur
    \item \textbf{Précondition} : Le recruteur possède un token LinkedIn valide et les informations de l'offre
    \item \textbf{Déclencheur} : Envoi d'une requête POST à \texttt{/api/job-offers/linkedin/publish}
    \item \textbf{Scénario nominal} :
    \begin{enumerate}[leftmargin=*]
        \item Le recruteur envoie une requête POST avec les champs : titre, description, localisation, entreprise, et les headers \texttt{Authorization} (Bearer token) et \texttt{X-Linkedin-User-Urn}.
        \item Le système valide la présence et le format du token.
        \item Le système construit le texte du post LinkedIn.
        \item Le système tente de publier l'offre sur LinkedIn via l'API.
        \item Le système stocke l'offre (avec le texte LinkedIn) en base de données.
        \item Le système retourne un message de succès ou d'erreur selon la réponse LinkedIn, mais l'offre est toujours stockée.
    \end{enumerate}
    \item \textbf{Scénarios alternatifs} :
    \begin{itemize}
        \item Si le token est manquant ou invalide, le système retourne une erreur 400.
        \item Si la publication LinkedIn échoue, l'offre est tout de même stockée et l'erreur est signalée dans la réponse.
    \end{itemize}
\end{itemize}

\section{Structure des données}
\textbf{Entrée (JobOfferRequestDTO)}
\begin{itemize}[leftmargin=*]
    \item title : String (obligatoire)
    \item description : String (obligatoire)
    \item location : String (obligatoire)
    \item company : String (obligatoire)
\end{itemize}

\textbf{Sortie (JobOfferResponseDTO)}
\begin{itemize}[leftmargin=*]
    \item id : Long
    \item title : String
    \item published : boolean
\end{itemize}

\textbf{Entité persistée (JobOffer)}
\begin{itemize}[leftmargin=*]
    \item id : Long
    \item title : String
    \item description : String
    \item location : String
    \item company : String
    \item published : boolean
    \item linkedinPostText : String
\end{itemize}

\section{Règles de gestion}
\begin{itemize}[leftmargin=*]
    \item Le texte LinkedIn est formaté avec un emoji, le titre, l'entreprise, la localisation et la description.
    \item L'offre est toujours stockée, même si la publication LinkedIn échoue.
    \item Le token LinkedIn doit être fourni dans le header Authorization (format Bearer).
\end{itemize}

\section{Sécurité}
\begin{itemize}[leftmargin=*]
    \item L'endpoint \texttt{/api/job-offers/linkedin/publish} est ouvert mais nécessite un token LinkedIn valide.
    \item Les autres endpoints sont protégés par défaut.
\end{itemize}

\chapter{Cahier de Tests}

\section{Tests fonctionnels}
\begin{table}[H]
\centering
\begin{tabular}{|l|p{5cm}|p{5cm}|p{4cm}|}
\hline
\textbf{ID} & \textbf{Cas de test} & \textbf{Entrée / Précondition} & \textbf{Attendu} \\
\hline
TC01 & Publication réussie & Offre complète, token valide, URN valide & 200 OK, message succès, offre stockée \\
\hline
TC02 & Token manquant & Offre complète, pas de header Authorization & 400, message d'erreur explicite \\
\hline
TC03 & Token invalide & Offre complète, token mal formé & 400, message d'erreur explicite \\
\hline
TC04 & Publication LinkedIn échoue & Offre complète, token valide, LinkedIn renvoie une erreur & 400, message d'erreur, offre stockée \\
\hline
TC05 & Données manquantes & Un ou plusieurs champs manquants dans l'offre & 400, message d'erreur explicite \\
\hline
TC06 & Vérification du stockage & Après publication (succès ou échec LinkedIn) & Offre présente en base avec texte LinkedIn \\
\hline
TC07 & Format du texte LinkedIn & Offre complète & Texte LinkedIn formaté correctement en base \\
\hline
\end{tabular}
\caption{Tests fonctionnels principaux}
\end{table}

\section{Tests de sécurité}
\begin{table}[H]
\centering
\begin{tabular}{|l|p{5cm}|p{5cm}|p{4cm}|}
\hline
\textbf{ID} & \textbf{Cas de test} & \textbf{Entrée / Précondition} & \textbf{Attendu} \\
\hline
TS01 & Endpoint protégé & Accès à un endpoint autre que /linkedin/publish sans auth & 401 Unauthorized \\
\hline
TS02 & Endpoint ouvert & Accès à /linkedin/publish sans auth Spring & 200/400 selon la présence du token LinkedIn \\
\hline
\end{tabular}
\caption{Tests de sécurité}
\end{table}

\section{Tests techniques}
\begin{table}[H]
\centering
\begin{tabular}{|l|p{5cm}|p{5cm}|p{4cm}|}
\hline
\textbf{ID} & \textbf{Cas de test} & \textbf{Entrée / Précondition} & \textbf{Attendu} \\
\hline
TT01 & Performance & 100 publications en 1 minute & Pas d'erreur, temps de réponse < 2s \\
\hline
TT02 & Robustesse & LinkedIn indisponible & Offre stockée, message d'erreur approprié \\
\hline
\end{tabular}
\caption{Tests techniques}
\end{table}

\chapter{Démonstration du Fonctionnement Réel}

Cette section illustre le fonctionnement effectif du service à travers une séquence complète :

\section{1. Récupération des informations utilisateur LinkedIn}
\begin{figure}[H]
    \centering
    \includegraphics[width=0.8\textwidth]{linkedin_userinfo.png}
    \caption{Appel GET /v2/userinfo avec Bearer Token (succès)}
\end{figure}

\section{2. Publication d'une offre via l'API REST}
\begin{figure}[H]
    \centering
    \includegraphics[width=0.8\textwidth]{postman_publish.png}
    \caption{Requête POST /api/job-offers/linkedin/publish avec offre complète}
\end{figure}

\section{3. Résultat sur LinkedIn}
\begin{figure}[H]
    \centering
    \includegraphics[width=0.5\textwidth]{linkedin_result.png}
    \caption{Offre publiée sur le profil LinkedIn}
\end{figure}

\section{4. Réponse LinkedIn API (ugcPosts)}
\begin{figure}[H]
    \centering
    \includegraphics[width=0.8\textwidth]{linkedin_ugcPosts.png}
    \caption{Réponse 201 Created de l'API LinkedIn après publication}
\end{figure}

\section{Scénario démontré}
\begin{itemize}[leftmargin=*]
    \item L'utilisateur récupère son profil LinkedIn avec un token valide.
    \item Il soumet une offre d'emploi via l'API REST locale.
    \item L'offre est publiée sur LinkedIn et stockée en base.
    \item L'API LinkedIn retourne l'identifiant du post créé.
    \item Le résultat est visible sur le profil LinkedIn.
\end{itemize}

\section{Extrait de code de test d'intégration}
Le test suivant vérifie que l'offre est bien stockée même si la publication LinkedIn échoue (token factice) :

\begin{verbatim}
@Test
void publishOffer_LinkedInErrorButStored() throws Exception {
    JobOfferRequestDTO dto = JobOfferRequestDTO.builder()
            .title("Développeur Java")
            .description("CDI, 3 ans d'expérience")
            .location("Paris, France")
            .company("Nexotek RH")
            .build();
    mockMvc.perform(post("/api/job-offers/linkedin/publish")
            .header("Authorization", "Bearer dummy-token")
            .header("X-Linkedin-User-Urn", "urn:li:person:1234")
            .contentType(MediaType.APPLICATION_JSON)
            .content(objectMapper.writeValueAsString(dto)))
            .andExpect(status().isBadRequest())
            .andExpect(content().string(org.hamcrest.Matchers.containsString("Erreur LinkedIn")));
}
\end{verbatim}

\noindent
Ce test simule une publication avec un token invalide et vérifie que l'offre est tout de même stockée, conformément aux exigences du cahier des charges.

\chapter*{Conclusion}
\addcontentsline{toc}{chapter}{Conclusion}
\thispagestyle{plain}
Ce service permet d'automatiser la publication d'offres sur LinkedIn tout en assurant la traçabilité via le stockage en base. L'architecture modulaire facilite l'extension à d'autres plateformes ou l'ajout de nouvelles fonctionnalités.

\end{document}